Эта задача немного необычна --- в ней вам предстоит реализовать интерактивное взаимодействие с тестирующей системой. Это означает, что вы можете делать запросы и получать ответы в online-режиме. Обратите внимание, что ввод/вывод в этой задаче --- стандартный (то есть с экрана на экран). После вывода очередного запроса обязательно используйте функции очистки потока, чтобы часть вашего вывода не осталась в каком-нибудь буфере. Например, на С++ надо использовать функцию \texttt{fflush(stdout)}, на Java вызов \texttt{System.out.flush()}, на Pascal \texttt{flush(output)} и \t{stdout.flush()} для языка Python.

В этой задаче вам предстоит в интерактивном режиме угадать число $x$, которое загадала тестирующая система. Про загаданное число $x$ известно, что оно целое и лежит в границах от $1$ до $n$ включительно (значение $n$ известно заранее).

Вы можете делать запросы к тестирующей системе, каждый запрос --- это вывод одного целого числа от $1$ до $n$. Есть два варианта ответа тестирующей системы на запрос:

\begin{itemize}
\item строка <<\texttt{<}>> (без кавычек), если загаданное число меньше числа из запроса;
\item строка <<\texttt{>=}>> (без кавычек), если загаданное число больше либо равно числу из запроса.
\end{itemize}

В случае, если ваша программа наверняка угадала нужное число $x$, выведите строку вида <<\texttt{! x}>>, где $x$ --- это ответ, и завершите работу своей программы.

Вашей программе разрешается сделать не более $25$ запросов.

\InputFile

Для чтения ответов на запросы программа должна использовать стандартный ввод.

В первой строке входных данных будет содержаться целое положительное число $n$ ($1 \le n \le 10^6$) --- максимально возможное число, которое может быть загадано.

В следующих строках на вход вашей программе будут подаваться строки вида <<\texttt{<}>> и <<\texttt{>=}>>. $i$-я из этих строк является ответом системы на ваш $i$-й запрос. После того, как ваша программа угадала число, выведите <<\texttt{! x}>> (без кавычек), где $x$ --- это ответ, и завершите работу своей программы.

Тестирующая система даст вашей программе прочитать ответ на запрос из входных данных только после того, как ваша программа вывела соответствующий запрос системе и выполнила операцию \texttt{flush}.

\OutputFile

Для осуществления запросов программа должна использовать стандартный вывод.

Ваша программа должна выводить запросы --- целые числа $x_i$ ($1 \le x_i \le n$) по одному в строке (не забывайте выводить <<\textit{перевод строки}>> после каждого значения $x_i$). После вывода каждой строки программа должна выполнить операцию \texttt{flush}.

Каждое из значений $x_i$ обозначает очередной запрос к системе. Ответ на запрос программа сможет прочесть из стандартного ввода. В случае, если ваша программа угадала число $x$, выведите строку вида <<\texttt{! x}>> (без кавычек), где $x$ --- ответ, и завершите работу программы.

\SAMPLES
